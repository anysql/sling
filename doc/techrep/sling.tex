\documentclass[11pt,a4paper]{article}
\usepackage[hyperref]{acl2017}
\usepackage{url}
\usepackage{times}
\usepackage{latexsym}
\usepackage{amsmath}
\usepackage{float}

\floatstyle{boxed}
\restylefloat{figure}

\aclfinalcopy

% type user-defined commands here

\title{SLING: A framework for frame semantic parsing}

\author{
Michael Ringgaard \\ Google Inc. \\ {\tt ringgaard@google.com} \\\And
Rahul Gupta \\ Google Inc. \\ {\tt grahul@google.com} \\
}

\date{September 13, 2017}

\begin{document}
\maketitle

\begin{abstract}
This technical report describes SLING, which is a framework for building
parsers for annotating text with semantic frames.
The parser is a general transition-based frame semantic parser using
bi-directional LSTMs for input encoding and a Transition Based Recurrent
Unit (TBRU) for output decoding.
It is a jointly trained model using only the text tokens as input and the
transition system has been designed to output frame graphs directly without
any intervening representation.
The SLING framework includes an efficient and scalable frame store
implementation as well as a neural network JIT compiler for fast parsing at
runtime.
The SLING framework is implemented in C++ and it is available for download
on GitHub.
\end{abstract}

\section{Introduction}

Recent advances in machines learning has enabled us to use deep learning to
train recurrent multi-level neural network classifiers. This has allowed us to
reconsider how to best design and implement systems for natural language
understanding (NLU).

Traditionally, natural language processing (NLP) systems have used a cascade of
NLP modules for syntatic and semantic annotation of text.
These systems typically first do part-of-speech (POS) tagging, followed by
constituency or dependency parsing for syntactic analysis.
Using the POS tags and parse trees as feature inputs, you could then do other
NLP tasks like mention chunking, entity type tagging, coreference resolution,
semantic role labeling (SRL), etc.

Pipelined systems often use a simple 1-best feed forward strategy \cite{finkel2006}
where they greedily take the best output of each stage and pass it on to the
next stage.
While it is simple to build such an architecture, errors can accumulate
throughout the pipeline making it much harder for the system to perform
correctly, e.g. F1 on SRL drops by more than 10\% when going from gold to
automatic parse trees \cite{toutanova2005}.

Often, applications do not need the intermediate annotations produced
by the lower levels in the NLP stack, so it would be preferable if these systems
could be trained jointly to optimize an objective based on the output
annotations needed for a particular application.

Classic NLP systems are also often trained using linear classifiers.
Because of the simplicity of these, very complex and elaborate feature
extraction with many different types of features as well as many kinds
of combinations of these features are often required for good accuracy.
With deep learning you can use embeddings, multiple layers, and recurrent
network connections instead, and this will often remove the need for complex
feature designs. The internal learned representations in the hidden layers of
the models can be used instead of the hand-crafted intermediate representations
in pipelined systems.

We have used DRAGNN \cite{dragnn} and Tensorflow \cite{tensorflow} for training the
parser.

\section{Frame semantics}

While frames in SLING are not tied to any particular frame semantic theory or
knowledge ontology, they are inspired by \emph{frame semantics}, which is a
theory of linguistic meaning developed by Charles Fillmore \cite{fillmore1982}.
Frame semantics connects linguistic semantics to encyclopedic knowledge and the
central idea is that understanding the meaning of a word requires access to all
the essential knowledge that relates to that word. A word \emph{evokes} a frame
repesenting the specific concept it refers to.

A semantic frame is a set of facts that specify "characteristic
features, attributes, and functions of a denotatum, and its characteristic
interactions with things necessarily or typically associated with it." \cite{alan2001}.
A semantic frame can also be vieved as a coherent structure of related concepts
that are related such that knowledge of all of them is required to have
complete knowledge of any one.

Frame semantics is not only for individual concepts, but can be generalized
to phrases, entities, constructions, and other larger and more complex linguistic
and ontological units. Semantic frames can be also used in information modeling
for constructing knowledge bases of world knowledge and common sense, and frame
semantics can also form the basis for reasoning about metaphors, metonymy,
actions, perspective, etc.

At a more technical level, a SLING frame consists of a list of slots, where each
slot has a name (role) and a value. The slot values can be literals like numbers
and strings, or links to other frames. The frames essentially form a graph where
the frames are the (typed) nodes and the slots are the labeled edges. A frame
graph can also be viewed as a feature structure \cite{carpenter2005} and
unification can be used for induction of new frames from existing frames.
Sometimes it is also useful to use frames for representing more basic data
structures like a C struct with fields, a JSON object, or a record in a
database.

SLING frames live inside a \emph{frame store}. A store is a container that
tracks the all the frames that have been allocated in the store, and serves as a
memory allocation arena for the allocated frames. When making a new frame, you
specify the store that the frame should be allocated in. The frame will live in
this store until its store is deleted or the frame is garbage collected because
there are no references to it.\footnote{See the \href{https://github.com/google/sling/blob/master/frame/README.md}{SLING Guide}
for a detailed description of the SLING frame store implementation.}

\begin{figure*}[t]
  \begin{verbatim}{
  :/s/document
  /s/document/text: "John hit the ball"
  /s/document/tokens: [
    {/s/token/text: "John" /s/token/start: 0  /s/token/length: 4},
    {/s/token/text: "hit"  /s/token/start: 5  /s/token/length: 3},
    {/s/token/text: "the"  /s/token/start: 9  /s/token/length: 3},
    {/s/token/text: "ball" /s/token/start: 13 /s/token/length: 4}
  ]
  /s/document/mention: {
    :/s/phrase /s/phrase/begin: 0
    /s/phrase/evokes: {=#1 :/s/person }
  }
  /s/document/mention: {
    :/s/phrase /s/phrase/begin: 1
    /s/phrase/evokes: {
      :/pb/hit-01
      /pb/arg0: #1
      /pb/arg1: #2
    }
  }
  /s/document/mention: {
    :/s/phrase /s/phrase/begin: 3
    /s/phrase/evokes: {=#2 :/s/consumer_good }
  }
}\end{verbatim}
  \caption{The text "John hit the ball" in SLING frame notation. The document
  itself is represented by a frame that has the text, an array of tokens and
  the mentions that evoke frames. There are three frames: a person frame (John),
  a consumer good frame (bat) and a hit-01 frame. The hit frame has the person
  frame as the agent (arg0) and the ball frame as the object (arg1).}
\end{figure*}

\section{Transition system}

In the parsing litteratur, \emph{transition systems} have been
used to construct dependency parse trees from a sequence of state-action pairs
$(s_i,a_i)$. A transition system takes a state $s_i$ and an action $a_i$ and
constructs a new state $s_{i+1}$. This allows you to build a tree structure by
predicting a sequence of actions. For example, the \emph{arc-standard}
transition system \cite{nivre2006} uses {\bf SHIFT}, {\bf LEFT-ARC(label)}, and
{\bf RIGHT-ARC(label)} actions to construct a dependency parse tree from a
sequence of these actions.

We are using the same core idea to construct a frame graph where frames can be
evoked by phrases in the input. Instead of using a stack, we have an
\emph{attention buffer}, which keeps track of the most salient frames in the
discourse. The actions in the transition system both builds up the frame graph
as well as maintaining the attention buffer.

\begin{itemize}
  \item {\bf SHIFT} -- Skips the next input token. Only valid when not at the
        end of the input buffer.

  \item {\bf STOP} --- Signals that we have reach the end of the parse. This is
        only valid when at the end of the input buffer. Multiple STOP actions
        can be added to the transition sequence, e.g. to make all sequences in a
        beam have the same length.

  \item {\bf EVOKE(type, n)} -- Evokes frame of with type {\bf type} from
        the next {\bf n} tokens in the input. The new frame will become the
        center of attention.

  \item {\bf REFER(frame, n)} -- Makes a new mention the next {\bf n} tokens in
        the input of evoking an existing frame in the attention buffer. This
        frame will become the new center of attention.

  \item {\bf CONNECT(source, role, target)} -- Adds slot to {\bf source} frame
        in the attention buffer with name {\bf role} and value {\bf target}
        where {\bf target} is an existing frame in the attention buffer. The
        {\bf source} frame become the new center of attention.

  \item {\bf ASSIGN(frame, role, value)} -- Adds slot to {\bf source} frame in
        the attention buffer  with name {\bf role} and value {\bf value} and
        moves the frame to the center of attention.

  \item {\bf EMBED(target, role, type)} -- Creates a new frame with
        type {\bf type} and add a slot to it with name {\bf role} and value
        {\bf target} where {\bf target} is an existing frame in the attention
        buffer. The new frame becomes the new center of attention.

  \item {\bf ELABORATE(source, role, type)} -- Creates a new frame with type
        {\bf type} and adds a slot to an existing frame {\bf source} in the
        attention buffer with {\bf role} set to the new frame. The new frame
        becomes the new center of attention.
\end{itemize}

\section{Parser runtime}

Myelin is a neural network JIT compiler... TBD

\section{OntoNotes corpus}

Descibe how the OntoNotes-based training and testing corpus was constructed...

\section{Evaluation}

An annotated document consists of a number of connected frames and as well as
phrases, i.e. token spans, evoking these frames. We evaluate the accuracy of the
annotations by comparing the generated frames with the gold standard frame
annotations from the evaluation corpus.

The documents are matched by constructing a virtual graph where the document
is the start node. The document node is then connected to the spans and the
spans are connected to the frames that the spans evoke. This graph is then
extended by following the frame-to-frame links from the roles. Accuracy is
computed by aligning the golden and predicted graphs and computing precision,
recall, and F1. The accuracy is computed for spans, frames, types, roles, and
labels.

An annotated document consists of a number of connected frames and phrases, i.e.
token spans, evoking the frames. We evaluate the accuracy of the annotations
by comparing the generated frames with the gold standard frame annotations from
the corpus.

\section{Furture work}

% Cascading action classifier to speed up computations of logits... TBD

% Knowledge-based inference... TBD

\section{Acknowledgments}

\bibliography{sling}
\bibliographystyle{acl_natbib}

\end{document}

